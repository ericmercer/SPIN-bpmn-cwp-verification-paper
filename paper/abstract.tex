Remote health-care that integrates human and machine intelligence for patient monitoring is an active area of research in health IT systems. These systems must take extra precautions to ensure safety since the patient is not in the direct supervision of the medical provider. This paper details the application of model checking to a remote COVID-19 patient monitoring system to prove the system ensures patient safety. The requirements for patient safety are formalized as a cognitive work problem that is translated to Linear Temporal Logic suitable for the SPIN model checker. A cognitive work problem is a computationally independent declarative model stating what a system must accomplish. The system itself is modeled with the Business Process Modeling Notation to capture the coordinated asynchronous and remote behaviors of the patient at home, the artificial intelligence in the cloud, and the medical providers. These two models with behavior models for vitals etc. are given to SPIN. SPIN proves the system preserves patient safety as defined in the cognitive work problem. This verification result complements conventional software evaluations and increases the safety and acceptance of such remote health IT systems.


