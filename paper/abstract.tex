Remote health-care that integrates human and machine intelligence for patient monitoring is an active area of research. These systems must take extra precautions for safety since the patients are not in the direct supervision of medical providers. This paper details the application of model checking to the Bionous \phware\ remote patient monitoring system to prove it preserves patient safety. Patient safety requirements are formalized in a cognitive work problem that is translated to Linear Temporal Logic properties. A cognitive work problem (CWP) is a computationally independent model stating what a system must accomplish. In this example, the system must take action on risk awareness to enhance patient safety, so the CWP defines risk awareness and requisite decisions given the current risk. The \phware\ workflow is translated to Promela to model the asynchronous behaviors of the patient at home, the artificial intelligence in the cloud, and the clinicians. The LTL and Promela models with added behaviors for patient severity are given to the SPIN model checker to prove the system implements the cognitive work problem, meaning it acts appropriately in regards to risk awareness. This result is an important contribution to conventional evaluations and contributes to the assurance of patient safety in remote health IT.