Functional integration of human cognition and machine reasoning is an industry-wide problem for critical systems where failure risks human health or safety. The vast differences between human and machine performance properties combined with the distributed and often asynchronous nature of human-machine systems makes manual analysis unreliable. This paper introduces the use of cognitive work problems (CWPs) and model checking as a methodical means for function integration of people and computing and to prove whether the resulting design is correct in all sequence of states. The approach leverages standard state diagrams and workflows for user participation and then translates those for model checking. The CWP state diagram is translated to a set of linear temporal logic (LTL) properties, and the workflow models are translated into an equivalent Promela model. The SPIN model checker verifies whether or not the Promela model implements the LTL properties defined by the CWP. The entire process is illustrated on a case study of a remote patient monitoring system for the safety of COVID-19 home care patients. The CWP specifies the actionable risk awareness the system must establish and maintain. The SPIN results verified that the design of the monitoring system serves its medical purpose and preserves patient safety properties. This formalized model-based design analysis is general to many other critical systems and contributes an important addition to conventional evaluations to enhance their safety.

\begin{comment}
Remote health-care that integrates human and machine intelligence for patient monitoring is an active area of research. These systems must take extra precautions for safety since the patients are not in the direct supervision of medical providers. This paper details the application of model checking to the Bionous \phware\ remote patient monitoring system to prove it preserves patient safety properties. Patient safety requirements are formalized in a cognitive work problem that is translated to Linear Temporal Logic properties. A cognitive work problem (CWP) is a computationally independent model stating what a system must accomplish. In this example, the system must take action on risk awareness to enhance patient safety, so the CWP defines risk awareness and requisite decisions given the current risk. The \phware\ workflow is translated to Promela to model the asynchronous behaviors of the patient at home, the artificial intelligence in the cloud, and the clinicians. The LTL and Promela models with added behaviors for patient severity are given to the SPIN model checker to prove the system implements the cognitive work problem, meaning it acts appropriately in regards to risk awareness. This result is an important contribution to conventional evaluations and contributes to the assurance of patient safety in remote health IT.
\end{comment}