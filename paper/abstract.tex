Functional integration of human cognition and machine reasoning is an industry-wide problem for systems where failure risks human health or safety. The differences between human and machine performance properties with the distributed nature of human-machine systems makes manual analysis unreliable. This paper introduces the use of cognitive work problems (CWPs) and model checking as a methodical means for function integration of people and computing and to prove whether the resulting design is correct. The approach leverages Business Process Models for user design participation and translates those to Promela for model checking. The CWP state diagram is translated to a set of linear temporal logic (LTL) properties. The SPIN model checker verifies whether or not the Promela model implements the LTL properties. The entire process is illustrated on a case study of a remote patient monitoring system for the safety of COVID-19 home care patients. The CWP specifies the \emph{actionable risk awareness} the system must establish and maintain. The SPIN results verify that the design of the monitoring system serves its medical purpose and preserves patient safety properties. This model-based design analysis is general to many other critical systems and contributes an important addition to conventional evaluations to enhance their safety.