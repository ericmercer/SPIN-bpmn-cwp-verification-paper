Functional integration of human cognition and machine reasoning is an industry-wide problem for systems where failure risks human health or safety. The differences between human and machine performance properties and the distributed nature of human-machine systems makes manual analysis impractical. This paper introduces a method of cognitive work problems (CWPs) and model checking as a rigorous means for function integration of people with computing, and proof whether the resulting design is correct. The approach leverages modeling standards for user participation. CWPs are modeled as finite state machines, and Business Process Modeling for system behavior. The CWP state diagram is translated to a set of linear temporal logic (LTL) properties, and the behavior model translated to Promela for model checking. The SPIN model checker verifies whether or not the Promela model implements the LTL properties. The entire method is illustrated as a case study of a remote patient monitoring system of COVID-19 home care patients. The CWP specifies the \emph{actionable risk awareness} the system must establish and maintain. The SPIN results verify that the design of the monitoring system serves its medical purpose to enhance patient safety. This model-based analysis appears general to other critical system domains and contributes importantly to current safety and reliability evaluations.