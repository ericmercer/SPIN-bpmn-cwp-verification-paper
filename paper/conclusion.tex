The CWP was coupled with workflow models to create a verification problem suitable for the SPIN model checker. This was accomplished by translating the CWP into an set of LTL properties that together express the same meaning as the CWP. The workflow models can be directly turned into equivalent models in Promela, the input language to SPIN, which is able to prove whether the workflows correctly implement the CWP. Such mechanized and automated reasoning is critical to assurance that the design of a complex distributed system is capable to establish and maintain \emph{actionable risk awareness}. 

The CWP defines thresholds of patient risk during home care and their appropriate actions, thereby making a clear connection from the verified design to its larger, societal purpose of safe care. Although the interaction between users and AI is only modestly complex, it strikes the balance needed to illustrate the full method with an example workflow that can be followed by readers without much BPMN familiarity. 
Once deployed, the complexity could easily increase if confirmation/dismissal of alerts is fed back to AI machine learning to increase alert accuracy.

Our approach's generality depends on discovery and modeling CWPs. Prior to model-checking CWPs their principles  were applied to human-computer integration for highly usable designs of integrated aircraft mission and maintenance scheduling \cite{workcentered}, joint U.S.-Russia maneuver planning for the \emph{International Space Station}  \cite{10.1145/1978942.1979311}, and health care coordination \cite{BERRY201615}. These CWPs define fundamental requirements for systems that must respond to events outside their direct control. The principles of CWP were recently adopted in the SysMl v.2 standard.

Model-checking scalability depends on workflows' number of asynchronous actors, synchronization points, etc. BPMN supports hierarchies of sub-processes for larger workflows, where CWPs may be defined for each; regardless, the case study is a key step towards automation. 

Our current research aims to automate CWP translation to LTL, and BPMN to Promela, so counter-examples from SPIN point directly to CWPs. This undertaking is a non-trivial engineering task as model checking requires domain expertise and the counter-examples produced by SPIN are infinite by virtue of LTL semantics. Ongoing efforts are refining the SPIN translations to include additional global states that correspond directly to CWP states and edges to simplify counter-example mapping. 

Other research under consideration could explore reuse of verified designs in watcher systems that monitor their deployed implementations. CWP models could also be used to derive efficiency measures, e.g., reducing the amount of activity that does not advance the CWP states could be a new approach to efficiency.  

The focus of conventional design methods is on software, which leaves important aspects of system success largely to chance. 
This paper's novel contribution is verifiable integration of human-computer teaming on cognitive tasks. 
The need is industry-wide. 
Expected benefits of verifiable integration include greater safety and reliability of systems for many other critical domains. 
