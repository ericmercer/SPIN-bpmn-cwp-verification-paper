Tele-health for remote patient care results in complex systems that merit careful design consideration to enhance patient safety while in remote care. The cognitive work problem, CWP, is an effective mechanism to declare the measure of risk to a patient in home care and what actions must be made based on such risk. This CWP can be coupled with workflow models of the asynchronous remote care system to create a verification problem suitable for the SPIN model checker.

The CWP can be directly turned into an set of LTL properties that together express the same meaning as the original CWP. The workflow models can be directly turned into equivalent models in Promela, the input language to SPIN, and together SPIN is able to prove the workflows implement the CWP. Such mechanized, and automated, reasoning is critical to augmenting assurance that a design of a complex distributed system fulfills its intended purpose. 

Model checking with SPIN requires domain expertise that is not readily available to most (any) designers. For the model-based design methodology detailed in this paper to be viable, the entire verification process must be automated. The translation to Promela of the workflows and the translation to LTL of the CWP must all take place without user intervention. That said, the user must still provide behavior for the input environment in some form. Such mechanization is ongoing.

The counter-examples produced by SPIN are not easy to understand without some domain expertise. The reason being is that the counter-examples are infinite and consist of complex predicate statements over the state of the system! These counter-examples can be greatly simplified, and rendered to a human readable form, with added state and logging in the Promela model. The counter-example can then be replayed by SPIN and viewed in terms of the model output rather than SPINs counter-example tracing output. Such work is ongoing.