The need to integrate and verify human-computer teams is industry-wide. Conventional practice focuses on the design of software, often to the exclusion of designing and integrating user activity. The practice results in system designs that are incomplete and leave key aspects of success to chance. The approach in this paper is a novel contribution to integrating and verifying human-computer teams as shown by the case-study on remote patient monitoring. 

The CWP is an effective way to define the risk to a patient during home care and what actions must be made based on such risk. It can be coupled with workflow models to create a verification problem suitable for the SPIN model checker. This is accomplished by translating the CWP into an set of LTL properties that together express the same meaning as the CWP. The workflow models can be directly turned into equivalent models in Promela, the input language to SPIN, and SPIN is able to prove the workflows implement the CWP. Such mechanized, and automated, reasoning is critical to augmenting assurance that a design of a complex distributed system fulfills its intended purpose. 

This case-study is not sufficient for general statements on the approach. CWPs have been applied to other types of cognitive tasks but without the model checking: health care coordination \cite{BERRY201615}, integrated mission and maintenance scheduling for aircraft \cite{workcentered}, and maneuver planning for the International Space Station \cite{10.1145/1978942.1979311}. Scalability depends on the workflow model and its number of asynchronous actors, synchronization points etc. The value of graphical workflow models diminishes with crowding, so BPMN supports hierarchies of sub-processes. In this practice, CWPs can be defined for verifying the sub-processes; regardless, the case study is a key step towards automation. 

Future work is to automate the entire process so that the underlying model checker is not directly exposed to the user and to directly connect the counter-examples from SPIN to the CWP rather than the LTL properties. This undertaking is a non-trivial engineering task as model checking requires domain expertise and the counter-examples produced by SPIN are infinite by virtue of LTL semantics. Ongoing efforts refine the SPIN translations to include additional global state that corresponds directly to CWP states and edges to simplify counter-example mapping.
