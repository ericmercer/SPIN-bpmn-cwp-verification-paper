Tele-health for remote patient care results in complex systems that merit careful
design consideration to protect patient safety while in remote care. The CWP is an effective way to define the risk to a patient during home care and what actions must be made based on such risk. 

The need to integrate and verify human-computer teams is industry-wide. Critical systems that depend on human-computer teaming have increased with the rapid adoption of powerful mobile devices. Conventional practice focuses on the design of software, often to the exclusion of designing and integrating user activity. The practice results in system designs that are incomplete, leaving key aspects of success to chance.

This CWP can be coupled with workflow models of the asynchronous remote care system to create a verification problem suitable for the SPIN model checker. This is accomplished by translating the CWP into an set of LTL properties that together express the same meaning as the original CWP. The workflow models can be directly turned into equivalent models in Promela, the input language to SPIN, and SPIN is able to prove the workflows implement the CWP. Such mechanized, and automated, reasoning is critical to augmenting assurance that a design of a complex distributed system fulfills its intended purpose. 

Model checking with SPIN requires domain expertise that is not readily available to most (any) designers. For the model-based design methodology detailed in this paper to be viable, the entire verification process must be automated. The translation to Promela of the workflows and the translation to LTL of the CWP must all take place without user intervention. That said, the user must still provide behavior for the input environment in some form. Such mechanization is ongoing.

The counter-examples produced by SPIN are not easy to understand without some domain expertise. The reason is that the counter-examples are infinite and consist of complex predicate statements over the state of the system! These counter-examples can be greatly simplified, and rendered to a human readable form, with added state and logging in the Promela model. The counter-example can then be replayed by SPIN and viewed in terms of the model output rather than SPINs counter-example tracing output. Such work is ongoing.

This paper with its case-study is not sufficient to make strong general statements about the scalability of the proposed approach as it depends on the many aspects on the workflow model such as the number of asynchronous actors, the level of non-determinism, and the number of synchronization points. The conjecture, based on this experience, is that graphical workflow models are only useful if they fit on one page, and that in and of itself helps scalability because of necessity, to make it fit on one page means that aspects of the workflow must be abstracted into subtasks to create hierarchy in the design. In this way, CWPs can be defined and verified for subtasks, and then soundly abstracted away in the higher level workflow models to manage complexity. It is worth noting that the CWP itself largely does not affect the scalability of the verification problem as the translation decomposes it into relative simple LTL properties.