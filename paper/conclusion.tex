The CWP effectively defines thresholds of risk to patients during home care and their appropriate actions. It establishes a clear connection between the verified system design and its larger, societal purpose of safe care.
The CWP was coupled with workflow models to create a verification problem suitable for the SPIN model checker. This is accomplished by translating the CWP into an set of LTL properties that together express the same meaning as the CWP. The workflow models can be directly turned into equivalent models in Promela, the input language to SPIN, and SPIN is able to prove whether a workflow implements a CWP. 
Such mechanized, and automated, reasoning is critical to assurance that the design of a complex distributed system fulfills its intended purpose. 

Our current research aims to automate CWP translation to LTL, and BPMN to Promela, so counter-examples from SPIN point directly to CWPs. This undertaking is a non-trivial engineering task as model checking requires domain expertise and the counter-examples produced by SPIN are infinite by virtue of LTL semantics. Ongoing efforts refine the SPIN translations to include additional global state that corresponds directly to CWP states and edges to simplify counter-example mapping.

Our approach's generality depends largely on discovery and modeling CWPs. Prior to model-checking CWPs their principles  were applied to human-computer integration for highly usable designs of integrated aircraft mission and maintenance scheduling \cite{workcentered}, joint U.S.-Russia maneuver planning for the \emph{International Space Station}  \cite{10.1145/1978942.1979311}, and health care coordination \cite{BERRY201615}. CWPs relate these technology applications to their larger societal purpose. The principles of CWP were recently adopted in the SysMl v.2 standard.

Model-checking scalability depends on workflows' number of asynchronous actors, synchronization points, etc. For larger workflows BPMN supports hierarchies of sub-processes, where CWPs may be defined for each; regardless, the case study is a key step towards automation. 

The focus of conventional design methods is on software, leaving the success of important user behavior largely to chance. This paper's novel contribution is verifiable integration of human-computer teaming on cognitive tasks. The need is industry-wide. Expected benefits of verifiable integration include greater safety and reliability of systems for many critical domains.
