The need to integrate and verify human-computer teams is industry-wide. Conventional practice focuses on the design of software, often to the exclusion of designing and integrating user activity. The practice results in system designs that are incomplete and leave key aspects of success to chance. The approach in this paper is a novel contribution to integrating and verifying human-computer teams as shown the case-study on tele-health for remote patient care 

The CWP is an effective way to define the risk to a patient during home care and what actions must be made based on such risk. It can be coupled with workflow models to create a verification problem suitable for the SPIN model checker. This is accomplished by translating the CWP into an set of LTL properties that together express the same meaning as the CWP. The workflow models can be directly turned into equivalent models in Promela, the input language to SPIN, and SPIN is able to prove the workflows implement the CWP. Such mechanized, and automated, reasoning is critical to augmenting assurance that a design of a complex distributed system fulfills its intended purpose. 

This paper with its case-study is not sufficient to make strong general statements about scalability as it depends on many aspects of the workflow model such as the number of asynchronous actors, the level of non-determinism, and the number of synchronization points. The conjecture, based on the experience here, is that graphical workflow models are only useful if they fit on one page. So to make workflows fit on one page means that aspects must be abstracted into subtasks to create hierarchy. In this way, CWPs can be defined and verified for subtasks, and then soundly abstracted away in the higher level workflow models. 

Future work is to automate the entire process so that the underlying model checker is not directly exposed to the user and to directly connect the counter-examples from SPIN to the CWP rather than the LTL properties. This undertaking is a non-trivial engineering task as model checking requires domain expertise and the counter-examples produced by SPIN are infinite by virtue of LTL semantics. Ongoing efforts refine the SPIN translations to include state that corresponds directly to CWP states and edges to simplify counter-example mapping.
