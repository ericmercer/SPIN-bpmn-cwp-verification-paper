The object state is defined by the physician orders (\texttt{orders}), the actual severity of the patient on the 4-point scale (\texttt{sevLvl}), the level of care the patient is able to maintain from themselves or a care giver while at home (\texttt{careCapLvl}), and the tending severity conjectured from data remotely being monitored between exams (\texttt{trndSevLvl}). The orders are modeled with an enumerated type in Promela.
%
{\small
\begin{lstlisting}[style=myPromela]
mtype = {homeCare, admission, discharge}
mtype orders = homeCare
\end{lstlisting}
}
%
\noindent The enumeration values have their intuitive meaning for the prescribed intervention. Initial values are based on the state transition diagram in the next section.\footnote{Initial values do affect the correctness of the model and each possible initial object state is verified.}

The patient severity, care capability, and trending severity are modeled as bytes since they have a limited range based on the 4-point scale.
%
{\small
\begin{lstlisting}[style=myPromela]
byte sevLvl = 0
byte careCapLvl = 2
byte trendSevLvl = 0
\end{lstlisting}
}
%
\noindent The initial \texttt{careCapLvl} value is set at the point where the patient requires admission to the hospital according to the DRG ratings. It is a constant value and does not change in the model.

The meaning of the relevant care states with the associated risks patients can occupy are computed from the conditions on the incoming ($I$) and outgoing ($O$) transitions for each state as $(\bigwedge_{i \in I}\ i) \wedge \neg(\bigvee_{o \in O}\ o)$. The meaning of a state then is that all the conditions on the incoming transitions hold while none of the conditions on the outgoing transitions hold. Consider the \emph{Pt in appropriate home care} state in \figref{fig:cwp}. The conjunction of the incoming transitions require
%
\[
  \mathtt{sevLvl} < 2 \wedge \mathtt{sevLvl} \le \mathtt{careCapLvl} \wedge \mathtt{orders} = \mathtt{homeCare}
\]
%
The negated disjunction of the outgoing conditions require
%
\[
  \mathtt{orders} \neq 2 \wedge \mathtt{tndSevLvl} \le \mathtt{careCapLvl}
\]
%
\noindent These are captured in a macro-definition in Promela.
%
{\small
\begin{lstlisting}[style=myPromela]
#define ptInAppropriateHomeCareState
(
      (    (! isRequiresHospital(sevLvl))
        && isHomeCare(orders)                                      
        && isWithinCareCapability(sevLvl)
        && isWithinCareCapability(trndSevLvl)
      )
)
\end{lstlisting}
}
%
\noindent Each of the sates in \figref{fig:cwp} are defined similarly.

\subsection{System Properties}
There are two system level LTL properties defined for the CWP. The first is that no object state exists that is not covered by one of the states in the CWP (e.g., the CWP states are universally inclusive); and the second is that the goal states exists. The first property is given by the following LTL formula.
%
{\small
\begin{lstlisting}[style=myPromela]
#define inAState  HospitalState
               || ptInAppropriateHomeCareState
               || ptInElevatedRiskHomeCareState
               || ptDischargedState
               || ptExpiredState
ltl alwayInAState {(always (inAState))}
\end{lstlisting}
}
%
\noindent It is a simple invariant on the disjunction of the states. The object state must \emph{always} be in some known state throughout the entirety of the workflow model.

The second system level property looks for executions ending in the goal states. The goal states for the CWP in \figref{fig:cwp} have no outgoing transitions and are \emph{Pt expired} or \emph{Pt discharged}.
%
{\small
\begin{lstlisting}[style=myPromela]
#define fair
(
  (
    (eventually 
      (ptDischargedState || ptExpiredState))
  )
)
ltl fairPathExists {(always (! fair))}
\end{lstlisting}
}
%
\noindent Here \texttt{fair} is an eventuality that should exist in the workflow. The property to verify is expressed as an invariant that \emph{should not hold} in the workflow (e.g., it should result in a verification error). 

The very nature of LTL makes existential properties awkward to express in that they are accomplished by writing a property that should not hold with the counter-example showing where it does not hold as a \emph{witness} to its existence. Claiming that the eventuality \texttt{fair} never holds means that if it does hold (e.g., exists), then the model checker will find a witness attesting as much.

The term \emph{fair} is a reference to the over-approximating nature of the workflow model. Indeed, in considering the workflow in \figref{fig:bpmn}, it is possible that a patient is never discharged or that the patient never expires. Such behavior is \emph{not fair} because it is not emblematic of the real world. As such, the property proves the existence of paths in the workflow that end in the goal states, and this same property is used later to restrict verification to only those paths that end in one of goal states thereby excluding from considering in verification any infinite workflow behaviors where the patient never recovers or expires.

\subsection{State Properties}
After the system level properties, there are three properties checked at each state. The first, is that the state exists somewhere in the model; the second is that the object is never in the state at the same time it is in another state (e.g., any one state is mutually exclusive to the other states); and the third is that only transitions allowed by the CWP are taken from the state. These are shown by example with the \emph{Pt in appropriate homecare state}.

The first property proves the existence of the state somewhere in the workflow model.
%
{\small
\begin{lstlisting}[style=myPromela]
ltl ptInAppropriateHomeCareExists 
{(fair implies 
  (always (! ptInAppropriateHomeCareState)))}
\end{lstlisting}
}
%
\noindent As before, \texttt{ptInAppropriateHomeCareExists} should fail verification because the counter-example is the witness that it does exist. Notice the use of \texttt{fair} in the implication in the property. If the left side of the implication is false, meaning that the workflow path being considered is not fair (e.g., the patient never recovers or expires), then the implication is satisfied by definition (i.e., it is vacuously true); however, if it is a fair path, then that path is checked for the existence of the state. This property should fail, and in doing so, produce a witness to the existence of the state on some fair path in the model.

The second property proves the state is mutually exclusive to the other states in the CWP.
%
{\small
\begin{lstlisting}[style=myPromela]
ltl ptInAppropriateHomeCareMutex {
  ( 
    always (
      ptInAppropriateHomeCareState implies (
        (
              (! HospitalState)
          && ptInAppropriateHomeCareState
          && (! ptInElevatedRiskHomeCareState)
          && (! ptDischargedState)
          && (! ptExpiredState)
        )
      )
    )
  )
}
\end{lstlisting}
}
%
\noindent Using the implication is safe because it is known that the state exists on some path in the system, and therefore, the implication cannot be vacuously true. When the system is in the \texttt{ptInAppropriateHomeCareState} state, it cannot be in any of the other CWP states. As before, the property must always hold in every state allowed by the workflow.

The third property proves that only transitions allowed by the CWP are implemented by the workflow.
%
{\small
\begin{lstlisting}[style=myPromela]
ltl ptInAppropriateHomeCareEdges {
  (
    fair implies (
      always (
        ptInAppropriateHomeCareState implies (
          ptInAppropriateHomeCareState until (
                ptInElevatedRiskHomeCareState
            ||  ptDischargedState
          )
        )
      )
    )
  )
}
\end{lstlisting}
}
%
\noindent Here the workflow stays in the indicated state \emph{until} it transitions to one of the two successor states allowed by the CWP. Those states are either \emph{Pt at elevated risk in home care} or \emph{Pt discharged}. The restriction to only fair paths in the workflow is necessary because if the path is not fair, then a transition to one of the two successors never takes place as the patient resides in the \texttt{ptInAppropriateHomeCareState} state forever violating the \emph{until} property. As before, it is an invariant that must hold on any fair path. 

The two system properties with the three properties for each of the CWP states makes a total of seventeen properties that must hold in order to claim that the workflow satisfies the CWP.