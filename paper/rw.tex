It is possible to translate live sequence charts to LTL \cite{KUMAR, KUMAR2009137}. These translations result in large intractable formulas whereas the work here creates several connected small formulas that are easy for SPIN. UML modeling uses \emph{synchronous observer automata} to encode and verify safety properties \cite{8906967}. The CWP can be thought of as a synchronous observer, and it is possible to express it in SPIN for safety verification, but that complicates existential properties and may result in a larger state space. Other work verifies that UML state diagrams implement their associated activity diagrams (workflows) with the NuSMV model checker \cite{7436156}. The workflows are turned into LTL just opposite of the work here.

There is some work related in translating models in the \emph{Business Process Execution Language} (BPEL) to Promela \cite{bpelToPromela}. The semantics are different and is limited in scope to web-services. BPMN choreographies have been modeled in Promela and verified with SPIN for deadlock, but choreographies ignore workflows and only model message sequencing \cite{choreography}. The translation of BPMN to Promela is inspired by existing methods for turning Petri Nets into equivalent Promela models \cite{petrinetToPromela, petrinetInspiration}. These however do not include data. The translation in this paper is also based off of early prototype translations of BPMN to Promela using message channels for synchronization \cite{bpmn2promela} whereas the translation here uses global variables to mitigate state explosion.