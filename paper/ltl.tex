The finite state machine is a recognized standard in software development for specifications, but it is not so for specification in model checking.
Temporal logic is the recognized standard for model checking specifications, and more specifically for SPIN, it is LTL, which is a predicate calculus that can specify temporal orderings.
As such, the CWP is translated to LTL for use as the required property in model checking.
The model checking verifies that the \emph{object state}, or the state of the workflow that claims to implement the CWP, conforms to the transitions defined by CWP.

Atomic propositions are defined to indicate when an object state resides in any particular CWP state.
Each is computed from incoming ($I$) and outgoing ($O$) transitions on each state as $(\bigwedge_{i \in I}\ i) \wedge \neg(\bigvee_{o \in O}\ o)$: for a state, the common conditions on the incoming transitions hold and none of the conditions on the outgoing transitions hold.
These atomic propositions must be mutually exclusive, as defined later, to pass verification.

Consider a simplified definition for the \emph{Pt in appropriate home care} state in \figref{fig:cwp}. The conjunction of the incoming transitions require
%
\[
\begin{array}{l}
  \mathtt{sevNeed} \le \mathtt{homeCare}\ \wedge\ \\
  \mathtt{orders} = (\mathtt{home}\ \mathrm{with}\ \mathtt{remote}\ \vee\ 
                    \mathtt{more} \vee \mathtt{adjustedRx})
\end{array}
\]
%
And the negated disjunction of the outgoing transitions require
%
\[
  \mathtt{orders} \neq \mathtt{discharge} \wedge \mathtt{tndSevNeed} \le \mathtt{homeCare}
\]
%
\noindent These together define when the object resides in the \emph{Pt in appropriate home care} state.

The first LTL property for the CWP is that its states must be universally inclusive meaning that the object state must reside, always, in some CWP state:
%
{\small
\begin{lstlisting}[style=myPromela,mathescape=true]
always(   InitState $\vee$ HospitalState 
        $\vee$ ptInAppropriateHomeCareState 
        $\vee$ ptInElevatedRiskHomeCareState 
        $\vee$ ptDischargedState $\vee$ ptExpiredState)
\end{lstlisting}
}
%
\noindent It is an invariant over the disjunction of CWP states.

The second LTL property is that the object state is able to reach either of the two terminal states in the CWP: \emph{Pt expired} or \emph{Pt discharged}.
These are states in the CWP with no outgoing transitions.
%
{\small
\begin{lstlisting}[style=myPromela,mathescape=true]
eventually(ptDischargedState $\vee$ ptExpiredState)
\end{lstlisting}
}
%
\noindent The property is named \texttt{fair} and is an eventuality that should exist in the workflow.
Its existence is verified with \texttt{\textbf{always} $\neg$fair} which witnesses the existence of \texttt{fair} as a counter example to the property.

\begin{comment}
The meaning of \emph{eventually} is that at some point in the future of the path currently being considered, there exists a state, where the expression \texttt{(ptDischargedState || ptExpiredState)} is true.
The property to verify is expressed as an invariant using the \emph{always} operator.
This invariant \emph{should not hold} in the workflow model (e.g., it should result in a verification error). 

The very nature of LTL makes existential properties awkward to express in that they are accomplished by writing a property that should not hold.
The counter-example to the property showing where it does hold is a \emph{witness} to the existential property.
Claiming that the eventuality \texttt{fair} never holds in any state in any path of the workflow means that if it does hold somewhere (e.g., exists), then the model checker will find a witness attesting as much.
Anytime a property is used on the left-hand side of an implication, then such an existential property should be verified.
More on this requirement later.

The term \emph{fair} is a reference to the over-approximating nature of the workflow model.
Indeed, in considering the workflow in \figref{fig:bpmn}, it is possible that a patient is never discharged or that the patient never expires.
Such a behavior is \emph{not fair} because it is not emblematic of the real world---patients eventually are discharged or expire.
As such, the property proves the existence of paths in the workflow that end in the goal states, and this same property is used later to restrict verification to only those paths that end in one of goal states thereby excluding from consideration, in verification, any infinite workflow behaviors where the patient never recovers or expires.
\end{comment}

The next set of three properties must hold for each CWP state:
(1) the state is reached by some object state of the workflow;
(2) the object state is only ever covered by exactly one CWP state;
and (3) only transitions allowed by the CWP are taken by the object state.
These are shown by example with the \emph{Pt in appropriate home care state}.

The first property is the existence of the state somewhere.
%
{\small
\begin{lstlisting}[style=myPromela,mathescape=true]
fair $\implies$ always $\neg$ptInAppropriateHomeCareState
\end{lstlisting}
}
%
\noindent The counter-example is the witness that the object state is reachable on some fair path.
Recall that a fair path ends in a terminal state.
The implication is not vacuous because the earlier property witnesses the existence of such a path.
Vacuity is avoided similarly in other properties as seen next.

The second property is the object state may only belong to exactly one CWP state.
%
{\small
\begin{lstlisting}[style=myPromela,mathescape=true]
always( ptInAppropriateHomeCareState $\implies$ 
  ( $\neg$InitState
  $\wedge$ $\neg$HospitalState 
  $\wedge$ ptInAppropriateHomeCareState 
  $\wedge$ $\neg$ptInElevatedRiskHomeCareState 
  $\wedge$ $\neg$ptDischargedState 
  $\wedge$ $\neg$ptExpiredState))
\end{lstlisting}
}
%
\noindent When the system is in the \emph{Pt in appropriate home care state} state, it cannot be in any other CWP state simultaneously.
The CWP would need to be modified with additional state if it fails to differentiate object states into unique CWP states.

The third property is that only transitions allowed by the CWP are implemented by the workflow.
%
{\small
\begin{lstlisting}[style=myPromela,mathescape=true]
fair $\implies$ 
  always(ptInAppropriateHomeCareState $\implies$ 
    (ptInAppropriateHomeCareState until
      ( ptInElevatedRiskHomeCareState
      $\vee$ ptDischargedState)))
\end{lstlisting}
}
%
\noindent Here the object state must remain \emph{until} it transitions to one of the allowed successor states in the CWP.
In this example those states are either \emph{Pt at elevated risk in home care} or \emph{Pt discharged}.

There are twenty total properties at the end of translation of the CWP to LTL.
Two for inclusiveness and fairness.
Eighteen are for the three properties required by each of the six states in the CWP: the states is reached, no other CWP state covers it, and only allowed transitions leave it.
All twenty properties must hold for an object state space to implement the CWP correctly.