\begin{figure}
  \begin{center}
    \begin{tabular}{c}
      \includegraphics[scale=0.33]{bpmn.png}
    \end{tabular}
  \end{center}
\caption{The \href{https://github.com/ericmercer/SPIN-bpmn-cwp-verification-paper/blob/main/26-Oct-2021-BPMN.png}{workflow model} for the \phware\ system.}
\label{fig:bpmn}
\end{figure}

As shown in Fig.2 the \phware\ system design was developed in the \emph{Business Process Model Notation} (BPMN) \cite{BPMN} using the The MATH tool-suite \cite{workflowmodel}.
Fig.2 shows different colored workflow lanes for clinicians, AI cloud services, and patient-caregivers. \figref{fig:bpmn}.
Applying and adapting BPMN to health care informatics is the focus of the recent \emph{BPM+} project in the Object Management Group.\footnote{Many thanks to Stephen White for reviewing the workflow and clarifying aspects of the BPMN semantics.}

Our method for functional integration was adapted from concurrent design engineering in aerospace, where specific tracks for multiple design disciplines coordinate their work by referencing a common physical design artifact \cite{10.1007/978-1-4471-1538-0_9}.
The technology-neutral CWP is pivotal for integrating cognitive work of joint human-computer teams. In Fig.2 the lanes served as the design tracks, and the CWP as the common artifact that their coordinated activity must transform to its goal state.
Each lane affects, and is affected by the CWP's states.

\figref{fig:bpmn} starts in the green-colored clinicians' lane at upper left of the workflow, when patients test positive.
Positive tests get exams (task 01). The red letters in the workflow correlate to transitions in the CWP in \figref{fig:cwp}.
Ambulatory care or hospital admission (task 03) corresponds to edges \textbf{B}.
A severity need that can be met by home care (task 02) satisfies edges \textbf{A} and \textbf{F} with workflows leading to orders for home care.
Discharge criteria include an additional test with a negative result following the recovery period (\textbf{C}).

Task 04 begins the patient-caregiver workflow, which then repeats recording vitals at provider-ordered times (task 05) until an exam results in patients being discharged (\textbf{C}), or the patient returns for an exam, or expires (\textbf{G}).The raw vitals data are transmitted by the smartphone application to the AI cloud services. 

The middle metallic grey lane represents the AI cloud services where machine learning operates on sensor data to assign numerical values to vital signs and analyze their trends (task 06).
A packet of values and trends over time is sent immediately to a specialized dashboard for clinicians.
An alert of elevated risk is added if the \emph{Personal Data Analyzer} reasons that a combination of current or trending values exceed home care capability (\textbf{D}).

The specialized dashboard for clinicians maintains risk awareness for each outpatient (right-top green lane).
Alerts (task 07a) skip any routine delay for the clinicians' attention.
The workflow and user interface are designed to preserve the clinicians' ultimate decision authority.
The doctor can quickly access the AI reasoning, confirm current or trending severity justifies the alert or revise it.
If alerts are confirmed, an urgent exam is ordered (task 08a).
An urgent exam may result in orders for ambulatory care or hospital admission (\textbf{E}).
When multiple patients have alerts, providers must prioritize them based on familiarity and professional judgement.
Providers may also decide trends can be controlled by adjusting orders, such as more frequent vitals reporting or changing medications (\textbf{F}).
If alerts are dismissed, a routine exam still may be due for scheduling (task 08b), or it may be scheduled to start soon.
All due exams, whether urgent or scheduled, go back to task 01. 
For non-alerts, providers review vitals when time permits (task 07b) and may order routine exams to be scheduled.
Alternatively, if clinical judgment deems appropriate, providers may order urgent exams.
The decision gate \textbf{Xor11} at far right is where all exam times are managed. All approaching exams flow to task 01 and held when all participants arrive.
