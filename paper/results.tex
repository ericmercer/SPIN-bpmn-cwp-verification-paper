The full Promela model, with the LTL properties for the CWP, is in a public Github repository \cite{repo}. The \emph{README.md} file in the repository summarizes its content and how to verify the model. The model is divided into three files: the CWP object state, the CWP LTL properties, and the BPMN workflow. A script, \emph{short-verify.sh}, combines the three files to create the Promela model and then runs SPIN on all the properties.

\begin{figure}[t]
  \begin{center}
    \begin{tabular}{c}
      \includegraphics[scale=0.50]{proof-digest.png}
    \end{tabular}
  \end{center}
\caption{Some of the verification results from the SPIN model checker.}
\label{fig:proof}
\end{figure}

The results for the state properties for \texttt{ptInElevatedRiskHomeCare} with the measured time verifying all properties is shown in \figref{fig:proof}. Verification takes around three to four minutes on an Intel Core i7 laptop with 16 Gb of RAM. It is not taxing the system in any way. 

All the existential properties (i.e., the properties ending with \emph{Exists}) result in an error. The error is the existential witness. All the other properties pass with no errors. The output of the script includes not just the error report but the coverage summary of the processes and properties. The first two entries pertain to the clinician and patient-caregiver processes. There should never be uncovered states in these processes. Uncovered states means that there are unreachable behaviors in the model---not good. The third entry is the property being verified. It is not unusual to have uncovered states here when there is no reported error since the error behavior is not found in the model---good.

This study is not sufficient to make strong general statements about the scalability of the proposed approach as it depends on the number or processes, the level of non-determinism in each process, and the number of synchronization points between each process. The conjecture, based on this experience, is that graphical workflow models are only useful if they fit on one page, and that in and of itself helps scalability because of necessity, making it fit on one page means that aspects of the workflow must be abstracted in subtasks to create hierarchy in the design. In this way, CWPs can be defined and verified for subtasks, and then soundly abstracted away in the higher level workflow models to manage complexity.