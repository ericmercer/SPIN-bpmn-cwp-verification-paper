\documentclass[conference]{IEEEtran}
\IEEEoverridecommandlockouts
\usepackage{cite}
\usepackage{amsmath}
\usepackage{amsthm}
\usepackage{amsfonts}
\usepackage{url}
\usepackage{comment}
\usepackage{listings}
\usepackage[usenames,dvipsnames,svgnames,table]{xcolor}

\definecolor{dkgreen}{rgb}{0,0.6,0}
\definecolor{mauve}{rgb}{0.58,0,0.82}
\definecolor{light-gray}{gray}{0.88}

\lstdefinestyle{myPromela}
{frame=none,
  basicstyle=\ttfamily,
  language=Promela,
  aboveskip=1mm,
  belowskip=1mm,
  showstringspaces=false,
  columns=flexible,
  numbers=none,
  numberstyle=\tiny\color{gray},
  commentstyle=\color{dkgreen},
  stringstyle=\color{mauve},
  breaklines=false,
  breakatwhitespace=true,
  tabsize=2,
  linewidth=2\linewidth
}

\hyphenation{op-tical net-works semi-conduc-tor}

\newcommand{\figref}[1]{Fig.~\ref{#1}}

\newcommand{\ie}{\textit{i.e.}}
\newcommand{\eg}{\textit{e.g.}}
\newcommand{\etal}{\textit{et al.}}
\newcommand{\etc}{\textit{etc.}}
\newcommand{\adhoc}{\textit{ad hoc}}

\begin{document}
\title{
  Model Checking A Remote Monitoring Heath IT System for Patient Safety
  \thanks{
    Someone?
  }
}

% author names and affiliations
% use a multiple column layout for up to three different
% affiliations
\author{
\IEEEauthorblockN{Eric Mercer}
\IEEEauthorblockA{Department of Computer Science\\
Brigham Young University\\
Provo, Utah}
\and
\IEEEauthorblockN{Keith Butler}
\IEEEauthorblockA{BPM+\\
Seatle, Washington}
}

% conference papers do not typically use \thanks and this command
% is locked out in conference mode. If really needed, such as for
% the acknowledgment of grants, issue a \IEEEoverridecommandlockouts
% after \documentclass

\maketitle

\begin{abstract}
Functional integration of human cognition and machine reasoning is an industry-wide problem for critical systems where failure risks human safety or health. The vast differences between human and machine performance properties combined with the inherently distributed and asynchronous nature of human-machine integration makes manual design analysis of any such systems difficult at best. This paper details the use of model checking as a means to systematize design analysis in a way that leverages the graphical workflows with which such systems are commonly modeled. The approach formalizes the functional intent of the system as a cognitive work problem (CWP) and then model checks that the workflow models implement the CWP. A CWP is a computationally independent model stating what a system must accomplish. It is translated into a set of linear temporal logic (LTL) properties that together have the same meaning as the CWP. The workflow models for the system are translated into an equivalent Promela model, and the SPIN model checker verifies whether or not the Promela model implements the LTL properties defined by the CWP. The entire process is illustrated in a case study on a remote patient monitoring system to verify that the system serves its medical purpose and preserves patient safety properties.  This formalized model-based design analysis is an important contribution to conventional evaluations and adds to the assurance of safety in critical systems.

\begin{comment}
Remote health-care that integrates human and machine intelligence for patient monitoring is an active area of research. These systems must take extra precautions for safety since the patients are not in the direct supervision of medical providers. This paper details the application of model checking to the Bionous \phware\ remote patient monitoring system to prove it preserves patient safety properties. Patient safety requirements are formalized in a cognitive work problem that is translated to Linear Temporal Logic properties. A cognitive work problem (CWP) is a computationally independent model stating what a system must accomplish. In this example, the system must take action on risk awareness to enhance patient safety, so the CWP defines risk awareness and requisite decisions given the current risk. The \phware\ workflow is translated to Promela to model the asynchronous behaviors of the patient at home, the artificial intelligence in the cloud, and the clinicians. The LTL and Promela models with added behaviors for patient severity are given to the SPIN model checker to prove the system implements the cognitive work problem, meaning it acts appropriately in regards to risk awareness. This result is an important contribution to conventional evaluations and contributes to the assurance of patient safety in remote health IT.
\end{comment}
\end{abstract}

\IEEEpeerreviewmaketitle

% \section{Introduction}
% There are limited tools for remote objective patient monitoring of diseases where patients may deteriorate quickly. Although tele-health is gaining acceptance for improving accessibility and reducing transmission rates \cite{10.1093/jamia/ocaa048,telehealth,10.1093/jamia/ocaa067}, the remote and asynchronous context of tele-health introduces the risk of a patient's conditions deteriorating before their physicians can be aware and intervene \cite{10.1097/ALN.0000000000003578}. Such risk always exists so the goal is to mitigate and manage that risk to enhance patient safety while in remote care. 

\begin{comment}
  Remote patient monitoring relies on clinicians, health IT, patients-caregivers, and other concurrent actors to each reliably perform various asynchronous tasks to coordinate patient care and safety \cite{remote,Aalam229}. Designing such systems becomes complex quickly because actors are outside the direct control of the system. These distributed and asynchronous characteristics make manual reasoning about functional integration and safety early in the design process very difficult; and yet, early in the design process is exactly the time to clearly establish the utility of the design in fulfilling its intended purpose. 
\end{comment}

This paper introduces a model-based approach for remote tele-health design that strives to manage and mitigate some aspects of the inherent risk through formal verification. The approach extends workflow models with conceptual models called \emph{cognitive work problems} (CWPs) for specification \cite{workflowmodel,workcentered,BERRY201615,chi2010}. When the health care work being modeled is substantially cognitive, tacit, and complex in nature, as in tele-health for remote patient monitoring, models, and especially graphical workflow models, are too complex for manual reasoning. This complexity stems from having to rely on clinicians, health IT, patients-caregivers, and other concurrent actors to each reliably perform various asynchronous tasks, all of which are not in the direct control of any one system, to coordinate patient care and safety \cite{remote,Aalam229}. These conceptual models complement and simplify workflow models by providing explicit specifications for the cognitive problem that must be solved. Formal verification can then be applied to prove whether or not a workflow model accomplishes its intended purpose early in the design process when change is less costly and intrusive.

This CWP augmented workflow modeling approach is presented here in a case study of Bionous \phware. \phware is a tele-health application that allows COVID-19 patients to care for themselves at home while being remotely monitored by clinicians. It provides a wearable ring or finger-clip to measure vital signs and communicate with a smartphone via Bluetooth. A smartphone application uploads the sensor data to AI cloud services for storage and analysis to increase the accuracy of values assigned to vitals \cite{Altschul2004PredictiveMI,10.2307/2984877,10.5555/1643031.1643047}. A web-based dashboard lets clinicians review vitals, trends, and alerts to prescribe appropriate interventions. 

The paper details the CWP for \phware, developed in conjunction with clinicians and other experts\footnote{Ann Marie Kimball, an internationally recognized  MD epidemiologist with specialty in emerging infectious disease, contributed significantly to the CWP}, that defines actionable risk awareness of patients as a finite state diagram with two dimensions: physiological events and actions for appropriate care of the patient. A novel contribution in this paper is the translation of the CWP model into a suite of \emph{Linear Temporal Logic} (LTL) properties that together exactly capture the meaning of the CWP. LTL is a first-order predicate calculus that includes the ability to reason about the temporal ordering of events \cite{10.5555/975331}. Any viable design for tele-health remote monitoring of COVID-19 patients must verify against each LTL property created by the CWP in order to claim it takes appropriate action relative to measured patient risk.

The paper then details the \phware\ workflow model. The model includes workflows for clinicians, AI cloud services, and patient-caregivers. A novel contribution in this paper is how the workflow models for \phware\ are translated into equivalent \emph{Process Meta-language} (Promela) models for the SPIN model checker \cite{spin}. This translation includes behavior models for the environment inputs such as patient vitals. The paper reports the SPIN verification results, and it worth noting that all the reported artifacts are available for independent certification \cite{repo}.

The intent of such a detailed case study is to show the viability of CWP augmented workflow modeling with model checking and to motivate the need for automated reasoning in any design environment. SPIN worked really well in this application in terms of performance and iterative design analysis. THe workflows and CWP revisions were easy to verify when posted, and when SPIN found violations, the accompanying counter-examples provided critical insights to the unexpected, and often overlooked, outcomes inherent in asynchronous interactions. These insights were absolutely necessary for those working on the workflows and those working on the CWP to arrive at the presented final forms. The experience of this integrated design process with the model checker makes clear that it is difficult to reliably reason about even simple concurrent systems without a verification tool.

\section{Conceptual Work Problem}
\begin{figure*}[t]
  \begin{center}
    \begin{tabular}{c}
      \includegraphics[scale=0.35]{cwp.png}
    \end{tabular}
  \end{center}
\caption{The CWP for remote COVID-19 patient care.}
\label{fig:cwp}
\end{figure*}

The purpose of \phware is remote monitoring that detects and analyzes seven vital signs to provide clinicians' with timely awareness of their outpatients' conditions and risk during home care, thereby supporting better decisions and enhancing patient safety. This purpose is, however, abstract and intangible. The CWP gives precise meaning to this purpose as a set of transformations on a complex object of work that is shared by activities in a distributed cognitive system.

To serve as an objective and operational definition of elevated risk due to home care, the work here adapts Medicare's 4-point severity-of-illness ratings \cite{Hornbrook2005OverviewOD,severity} and the CDC's Guidance on Home Care \cite{cdc}. The adaptation defines a \emph{Severity Level} less than two as mild, and a \emph{Home Care Capability Level} equal to two as typically appropriate. It then defines \emph{Elevated Outpatient Risk} from home care as either having a \emph{Trending Severity Level} or Severity Level greater than the Home Care Capability Level.

The CWP of actionable risk awareness was developed using these definitions as a finite state diagram, \figref{fig:cwp}, of the relevant care states patients can occupy, their associated risks, and the transition conditions among them. The transition conditions are based on the physician orders (\texttt{orders}), the current severity of the outpatient's condition assigned by a clinician's exam (\texttt{sevLvl}), the level of care at the outpatient's residence (\texttt{careCapLvl}), and the trending severity calculated from the remotely monitored longitudinal data as provided by
\phware\ cloud analytics (\texttt{trndSevLvl}). 

In the initial state (top) the patient has tested positive. The arc labeled \textbf{A} occurs when an exam shows symptoms of low severity and a provider orders home care with \phware. Outpatients remain in appropriate home care until either an exam finds discharge criteria met and ordered (\textbf{C}), or the trending severity level is greater than their care capability level (\textbf{D}). In such cases, patients are at an elevated risk in home care. They must not remain at an elevated risk in home care because there is a possible direct path to fatality (\textbf{G}). So a near future exam must either order admission (\textbf{E}), find their risk lower than what the analytics claimed, or make their risk lower with additional prescribed interventions (\textbf{F}). Patients who are admitted to hospital may eventually be discharged back to home care (\textbf{H}) or discharged directly (\textbf{C}). 

The declarative knowledge of the CWP specifies \emph{what} workflows solving the CWP must accomplish without depending on \emph{how} they do it. In this application, the CWP motivates the design, and then when the design is ready, the CWP is the property for formal verification that must be satisfied. 

The validation of the CWP for \phware\ is an interesting problem that is outside the scope of this paper but merits some discussion. Doctors reviewed the CWP using it as a kind of story-board whose rules at each arc are verified as appropriate care. One review noted that the CWP does not distinguish among causes of fatalities but concluded that any patient trending towards fatality must receive life-saving care regardless of co-morbidities, and so no change was required. The SPIN verification also served to validate the CWP and an example of that is given in \secref{sec:results}.



\section{Environment Model}
The model of the workflow and CWP object state require behaviors defined for all input. For example, the CWP object state is transformed over time by various distributed activities carried out by different actors. Some aspects of the object state are transformed directly by the system workflow while others are inputs on which a workflow makes decisions. These later inputs form the \emph{environment} in which the CWP exists. For verification, the behavior of these environment inputs must be modeled, and the verification results only hold relative to the behaviors considered by that environment model. The same is true for aspects of the workflow. Each is discussed below.

\subsection{CWP Environment Model}
In this model, all the data in the CWP object state are part of the environment, and as such, are inputs on which any workflow makes decisions. For example, the severity of the patient is established by an exam with a physician, but that severity is fundamentally a characteristic of the patient and the progression of the disease. The physician assesses the patient and ascribes a severity, but the patient, with their symptoms, are an input to the model. The same with the care capability level and the trending severity. These are values on which a workflow makes decisions regarding patient risk and appropriate intervention. 

There is, of course, a causal relationship between the decisions in a workflow and the resulting subsequent input. For example, it is normally expected that when a doctor orders a patient admitted to the hospital that at some point in the future the severity rating for the patient diminishes due to the increased level of intervention and care. Here is where modeling choices can limit the impact, and meaning, of any verification results as the verification only hold for the modeled input behavior.

This model of the CWP for remote patient care creates the weakest, that is to say, the least restrictive, environment model possible in which a workflow is able to be verified. That means the environment model includes behavior that exists in the real world (i.e., the causal relationships), and it includes behaviors that do not exist in the real world. The goal is to create a \emph{sound over-approximation} of feasible environment behavior meaning that it includes the real world behavior plus other behavior that is not possible in the real world. If a workflow verifies in the sound over-approximated environment, then by implication, that verification result holds in the real world since those behaviors are a subset of those considered for verification. 

The verification model assumes the \texttt{orders} from an examining physician correlate strongly with \texttt{severity}. A \texttt{severity} level below two does not require hospitalization according to the DRG scale while anything above two is the opposite.
%
{\small
\begin{lstlisting}[style=myPromela]
inline updateDoctorOrders(severity, orders) {
if
:: isRequiresHospital(severity) ->
  setHospital(orders)
:: else ->
  if
    :: (severity != 0) -> 
      setHomeCare(orders)
    :: else -> 
      setDischarge(orders)
  fi
fi
logOrders(orders)
}
\end{lstlisting}
}
%
\noindent The orders are deterministic based on the value of \texttt{severity}. The patient is always admitted to the hospital when \texttt{severity} is two or greater; otherwise, the patient is either discharged or prescribed continued care at home. In this model, a patient is only discharged if \texttt{severity} is 0 at the time of exam.

The model for the \texttt{severity} strongly correlates with \texttt{trendSeverity}.
%
{\small
\begin{lstlisting}[style=myPromela]
inline updatePatientSeverity(trendSeverity, 
  severity) {
  if
  :: isWithinCareCapability(trendSeverity) -> 
      if
      :: true -> setSeverity(severity, 0)
      :: true -> setSeverity(severity, 1)
      fi
  :: !isWithinCareCapability(trendSeverity) -> 
      setSeverity(severity, 2)
  :: true
  fi
  setSeverity(trendSeverity, severity)
  logSeverity(severity)
}
\end{lstlisting}
}
%
\noindent Promela semantics are non-deterministic, meaning that the model produces different outcomes from run to run. The model defines where that non-determinism takes place with choices to consider. Model checking considers all the ways possible to resolve the non-determinism to construct an exhaustive proof that a property holds in the model. 

The if-statement is one way to specify a point of non-determinism. In the model for \texttt{severity}, at the top level if-statement, the lone \texttt{:: true}, is an always enabled choice for the if-statement that leaves \texttt{severity} unchanged. The other choices are determined by the \texttt{trendSeverity} value with there being two choices when \texttt{trendSeverity} is within the care capability of the patient.

The final note is that the model synchronizes the value of \texttt{trendSeverity} after choosing a value for \texttt{severity}. The reasoning is that \texttt{severity} should only be updated at the time of an actual exam. As such, \texttt{trendSeverity} should reflect the outcome of since the two values are strongly correlated.

Of note is that \texttt{severity} may never be zero! It is possible for the model checker to resolve the non-determinism in such a way that \texttt{severity} is always one, simply always stays the same, etc. The result is that the patient may never be discharged and may never ever expire at the same time. Such behavior exists in the model but is not consistent with the real world. These are part of the over-approximation, and when necessary, are constrained out by changing the model or constraining what behaviors are considered in the model with a fairness property as discussed previously.

The model for \texttt{trendSeverity} is simple with only two non-deterministic choices that are independent of the actual \texttt{severity} of the patient. 
%
{\small
\begin{lstlisting}[style=myPromela]
inline updateSeverityTrend(trendSeverity) {
  if
  :: true -> 
    setWithinCareCapability(trendSeverity)
  :: true -> 
    setOutsideCareCapability(trendSeverity)
  fi

  logTrend(trendSeverity)
}
\end{lstlisting}
}
%
\noindent The amount that a patient is within or without the bounds of their care capability is not considered in this model; rather, \texttt{trendSeverity} is reduced to a Boolean proposition indicating if the case is one way or the other.

There is one final model that is part of of the environment and that is the model for patient mortality. 
%
{\small
\begin{lstlisting}[style=myPromela]
inline updatePatientMortality(trendSeverity, 
  severity) {
  if
  :: !isWithinCareCapability(trendSeverity) -> 
    setSeverity(severity, EXPIRED)
  :: !isWithinCareCapability(severity) ->
    setSeverity(severity, EXPIRED)
  :: true
  fi
  logSeverity(severity)
}
\end{lstlisting}
}
%
\noindent A patient may expire anytime the \texttt{severity} or \texttt{trendSeverity} is outside the bounds of the care capability.

\subsection{Workflow Environment Model}

The workflow makes decisions at different points based on the values of \texttt{alert}, \texttt{examType}, and \texttt{examTime}. Their values are updated by their associated tasks in the workflow, but how the values evolve over time is not specified. As such, they are modeled as inputs provided by the environment which are sampled when their associated tasks are activated.

The \texttt{alert} is only correlated with \texttt{trendSeverity} in that it is updated whenever the trend is updated, but aside from that, it is unconstrained; thus, it is able to be raised or not raised each time regardless of the value of \texttt{trendSeverity}. The \texttt{examType} is correlated with \texttt{trendSeverity} in that it can become urgent if the trend is outside the care capability level. The \texttt{examTime} is non-deterministic once it is scheduled meaning it randomly choses between leaving it scheduled and having the exam be now.

This non-determinism in \texttt{examTime} means that one behavior in the workflow model is that \texttt{examTime} is never \texttt{now} and is infinitely just \texttt{scheduled}. Such behavior is one of several behaviors that are part of the workflow model but not consistent with the real world in which the workflow exists.


% An example of a double column floating figure using two subfigures.
% (The subfig.sty package must be loaded for this to work.)
% The subfigure \label commands are set within each subfloat command,
% and the \label for the overall figure must come after \caption.
% \hfil is used as a separator to get equal spacing.
% Watch out that the combined width of all the subfigures on a 
% line do not exceed the text width or a line break will occur.
%
%\begin{figure*}[!t]
%\centering
%\subfloat[Case I]{\includegraphics[width=2.5in]{box}%
%\label{fig_first_case}}
%\hfil
%\subfloat[Case II]{\includegraphics[width=2.5in]{box}%
%\label{fig_second_case}}
%\caption{Simulation results for the network.}
%\label{fig_sim}
%\end{figure*}
%
% Note that often IEEE papers with subfigures do not employ subfigure
% captions (using the optional argument to \subfloat[]), but instead will
% reference/describe all of them (a), (b), etc., within the main caption.
% Be aware that for subfig.sty to generate the (a), (b), etc., subfigure
% labels, the optional argument to \subfloat must be present. If a
% subcaption is not desired, just leave its contents blank,
% e.g., \subfloat[].

% trigger a \newpage just before the given reference
% number - used to balance the columns on the last page
% adjust value as needed - may need to be readjusted if
% the document is modified later
%\IEEEtriggeratref{8}
% The "triggered" command can be changed if desired:
%\IEEEtriggercmd{\enlargethispage{-5in}}

% references section
% \bibliographystyle{IEEEtran}
% \bibliography{paper}
\end{document}


