\documentclass[conference]{IEEEtran}
\IEEEoverridecommandlockouts
\usepackage{cite}
\usepackage{amsmath}
\usepackage{amsthm}
\usepackage{amsfonts}
\usepackage{url}
\usepackage{comment}
\hyphenation{op-tical net-works semi-conduc-tor}


\newcommand{\figref}[1]{Fig.~\ref{#1}}

\newcommand{\ie}{\textit{i.e.}}
\newcommand{\eg}{\textit{e.g.}}
\newcommand{\etal}{\textit{et al.}}
\newcommand{\etc}{\textit{etc.}}
\newcommand{\adhoc}{\textit{ad hoc}}

\begin{document}
\title{
  Model Checking A Remote Monitoring Heath IT System for Patient Safety
  \thanks{
    Someone?
  }
}

% author names and affiliations
% use a multiple column layout for up to three different
% affiliations
\author{
\IEEEauthorblockN{Eric Mercer}
\IEEEauthorblockA{Department of Computer Science\\
Brigham Young University\\
Provo, Utah}
\and
\IEEEauthorblockN{Keith Butler}
\IEEEauthorblockA{BPM+\\
Seatle, Washington}
}

% conference papers do not typically use \thanks and this command
% is locked out in conference mode. If really needed, such as for
% the acknowledgment of grants, issue a \IEEEoverridecommandlockouts
% after \documentclass

\maketitle

\begin{abstract}
Remote health-care that integrates human and machine intelligence for patient monitoring is an active area of research. These systems must take extra precautions for safety since the patients are not in the direct supervision of medical providers. This paper details the application of model checking to the Bionous \phware\ remote patient monitoring system to prove it preserves patient safety. Patient safety requirements are formalized in a cognitive work problem that is translated to Linear Temporal Logic properties. A cognitive work problem (CWP) is a computationally independent model stating what a system must accomplish. In this example, the system must take action on risk awareness to enhance patient safety, so the CWP defines risk awareness and requisite decisions given the current risk. The \phware\ workflow is translated to Promela to model the asynchronous behaviors of the patient at home, the artificial intelligence in the cloud, and the medical providers. The LTL and Promela models with added behaviors for patient severity are given to the SPIN model checker to prove the system implements the cognitive work problem meaning it acts appropriately in regards to risk awareness. This result is an important contribution to conventional evaluations and contributes to the assurance of patient safety in remote health IT.
\end{abstract}

\IEEEpeerreviewmaketitle

% \section{Introduction}
% There are limited tools for remote objective patient monitoring of diseases where patients may deteriorate quickly. Although tele-health is gaining acceptance for improving accessibility and reducing transmission rates \cite{10.1093/jamia/ocaa048,telehealth,10.1093/jamia/ocaa067}, the remote and asynchronous context of tele-health introduces the risk of a patient's conditions deteriorating before their physicians can be aware and intervene \cite{10.1097/ALN.0000000000003578}. Such risk always exists so the goal is to mitigate and manage that risk to enhance patient safety while in remote care. 

\begin{comment}
  Remote patient monitoring relies on clinicians, health IT, patients-caregivers, and other concurrent actors to each reliably perform various asynchronous tasks to coordinate patient care and safety \cite{remote,Aalam229}. Designing such systems becomes complex quickly because actors are outside the direct control of the system. These distributed and asynchronous characteristics make manual reasoning about functional integration and safety early in the design process very difficult; and yet, early in the design process is exactly the time to clearly establish the utility of the design in fulfilling its intended purpose. 
\end{comment}

This paper introduces a model-based approach for joint human-computer teaming in remote tele-health 
that strives to manage and mitigate some aspects of the inherent risk through formal design verification. The contribution to functional integration is a computationally independent model of the \emph{cognitive work problem} (CWP) that a system must solve \cite{workflowmodel,workcentered,BERRY201615,chi2010}.  A CWP is a complex \emph{object of work} that can be shared by the activities of a distributed cognitive system to transform it from some initial state to the required goal state. The technical neutrality of this object makes it sharable by actors whose performance properties are vastly different; thus, the CWP provides a common basis to represent its state changes that can be performed by a joint team of people and computers. 

The contribution to design verification then re-uses a CWP as a new form of evaluation criterion for model-checking to prove the correctness of the resulting integrated system designs. The capability to integrate human cognition with computer reasoning and then verify the design correctness increases the safety for a wide range of critical systems where human well-being is jeopardized by leaving human-computing teaming to informal approaches \cite{remote,Aalam229}.

This CWP augmented-workflow modeling approach is presented here in a case study of Bionous \phware. \phware is a tele-health application that allows COVID-19 patients to care for themselves at home while remotely monitored by clinicians. It provides a wearable ring or finger-clip to measure vital signs and communicate with a smartphone via Bluetooth. A smartphone application uploads the sensor data to AI cloud services for storage and analysis to increase the accuracy of values assigned to vitals \cite{Altschul2004PredictiveMI,10.2307/2984877,10.5555/1643031.1643047}. A web-based dashboard lets clinicians review vitals, trends, and alerts to prescribe appropriate interventions. 

The paper details the CWP for \phware, developed in conjunction with clinicians and other experts,\footnote{Ann Marie Kimball, an internationally recognized MD epidemiologist in emerging infectious disease, provided medical advice} that defines actionable risk awareness of patients as a finite state diagram with two dimensions: physiological events and actions for appropriate care of the patient. This paper also details the translation of the CWP model into a suite of \emph{Linear Temporal Logic} (LTL) properties that together exactly capture the meaning of the CWP. LTL is a first-order predicate calculus that includes the ability to reason about the temporal ordering of events \cite{10.5555/975331}. Any viable design for tele-health remote monitoring of COVID-19 patients must verify against each LTL property created by the CWP in order to claim it takes appropriate action relative to measured patient risk.

The paper then describes the \phware\ workflow model. The model includes workflows for clinicians, AI cloud services, and patient-caregivers. This paper details how the workflow models for \phware\ are translated into equivalent \emph{Process Meta-language} (Promela) models for the SPIN model checker \cite{spin}. This translation includes behavior models for the environment inputs such as patient vitals. The paper reports the SPIN verification results. All the reported artifacts are available for independent certification \cite{repo}.

The intentions of such a detailed case study are to show the viability of CWP augmented workflow modeling with model checking and to motivate the need for automated reasoning in any design environment. SPIN worked really well in this application in terms of performance and iterative design analysis. The workflows and CWP revisions were easy to verify when posted, and when SPIN found violations, the accompanying counter-examples provided critical insights to the unexpected, and often overlooked, outcomes inherent in asynchronous interactions. These insights were absolutely necessary for those working on the workflows and those working on the CWP to arrive at the presented final forms. The experience of this integrated design process with the model checker makes clear that it is difficult to reliably reason about even simple concurrent systems without a verification tool.

% \section{Simple Example}
% \input{example}

% \section{System Model and Semantics}
% \input{system}

% \section{Contiguity Types}
% \input{contig} 

% \section{Filters}
% \input{filters}

% \section{Monitors}
% \input{monitors}

% \section{OSATE Integration for AADL}
% \input{aadl-agree-osate}

% \section{Related Work}
% \input{related-work}
% \section{Conclusion}
% The CWP was coupled with workflow models to create a verification problem suitable for the SPIN model checker. This was accomplished by translating the CWP into an set of LTL properties that together express the same meaning as the CWP. The workflow models can be directly turned into equivalent models in Promela, the input language to SPIN, which is able to prove whether the workflows correctly implement the CWP. Such mechanized and automated reasoning is critical to assurance that the design of a complex distributed system is capable to establish and maintain \emph{actionable risk awareness}. 

The CWP defines thresholds of patient risk during home care and their appropriate actions, thereby making a clear connection from the verified design to its larger, societal purpose of safe care. Although the interaction between users and AI is only modestly complex, it strikes the balance needed to illustrate the full method with an example workflow that can be followed by readers without much BPMN familiarity. 
Once deployed, the complexity could easily increase if confirmation/dismissal of alerts is fed back to AI machine learning to increase alert accuracy.

Our approach's generality depends on discovery and modeling CWPs. Prior to model-checking CWPs their principles  were applied to human-computer integration for highly usable designs of integrated aircraft mission and maintenance scheduling \cite{workcentered}, joint U.S.-Russia maneuver planning for the \emph{International Space Station}  \cite{10.1145/1978942.1979311}, and health care coordination \cite{BERRY201615}. These CWPs define fundamental requirements for systems that must respond to events outside their direct control. The principles of CWP were recently adopted in the SysMl v.2 standard.

Model-checking scalability depends on workflows' number of asynchronous actors, synchronization points, etc. BPMN supports hierarchies of sub-processes for larger workflows, where CWPs may be defined for each; regardless, the case study is a key step towards automation. 

Our current research aims to automate CWP translation to LTL, and BPMN to Promela, so counter-examples from SPIN point directly to CWPs. This undertaking is a non-trivial engineering task as model checking requires domain expertise and the counter-examples produced by SPIN are infinite by virtue of LTL semantics. Ongoing efforts are refining the SPIN translations to include additional global states that correspond directly to CWP states and edges to simplify counter-example mapping. 

Other research under consideration could explore reuse of verified designs in watcher systems that monitor their deployed implementations. CWP models could also be used to derive efficiency measures, e.g., reducing the amount of activity that does not advance the CWP states could be a new approach to efficiency.  

The focus of conventional design methods is on software, which leaves important aspects of system success largely to chance. 
This paper's novel contribution is verifiable integration of human-computer teaming on cognitive tasks. 
The need is industry-wide. 
Expected benefits of verifiable integration include greater safety and reliability of systems for many other critical domains. 

ACKNOWLEDGEMENTS

Earlier parts of this research were funded by:
AHRQ#1R011HS021233-01A1, ONC-#1051059, and  AFRL #F33615-03-2-6315.

% An example of a double column floating figure using two subfigures.
% (The subfig.sty package must be loaded for this to work.)
% The subfigure \label commands are set within each subfloat command,
% and the \label for the overall figure must come after \caption.
% \hfil is used as a separator to get equal spacing.
% Watch out that the combined width of all the subfigures on a 
% line do not exceed the text width or a line break will occur.
%
%\begin{figure*}[!t]
%\centering
%\subfloat[Case I]{\includegraphics[width=2.5in]{box}%
%\label{fig_first_case}}
%\hfil
%\subfloat[Case II]{\includegraphics[width=2.5in]{box}%
%\label{fig_second_case}}
%\caption{Simulation results for the network.}
%\label{fig_sim}
%\end{figure*}
%
% Note that often IEEE papers with subfigures do not employ subfigure
% captions (using the optional argument to \subfloat[]), but instead will
% reference/describe all of them (a), (b), etc., within the main caption.
% Be aware that for subfig.sty to generate the (a), (b), etc., subfigure
% labels, the optional argument to \subfloat must be present. If a
% subcaption is not desired, just leave its contents blank,
% e.g., \subfloat[].

% trigger a \newpage just before the given reference
% number - used to balance the columns on the last page
% adjust value as needed - may need to be readjusted if
% the document is modified later
%\IEEEtriggeratref{8}
% The "triggered" command can be changed if desired:
%\IEEEtriggercmd{\enlargethispage{-5in}}

% references section
% \bibliographystyle{IEEEtran}
% \bibliography{paper}
\end{document}


