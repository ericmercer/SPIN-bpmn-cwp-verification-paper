There are limited tools for remote objective patient monitoring of diseases where patients may deteriorate quickly. Although tele-health is gaining acceptance for improving accessibility and reducing transmission rates \cite{10.1093/jamia/ocaa048,telehealth,10.1093/jamia/ocaa067}, the remote and asynchronous context of tele-health introduces the risk of a patients' conditions deteriorating before their physicians can be aware and intervene \cite{10.1097/ALN.0000000000003578}. The goal of remote monitoring is to mitigate and manage that risk to enhance patient safety while in remote care. 

\begin{comment}
  Remote patient monitoring relies on clinicians, health IT, patients-caregivers, and other concurrent actors to each reliably perform various asynchronous tasks to coordinate patient care and safety \cite{remote,Aalam229}. Designing such systems becomes complex quickly because actors are outside the direct control of the system. These distributed and asynchronous characteristics make manual reasoning about functional integration and safety early in the design process very difficult; and yet, early in the design process is exactly the time to clearly establish the utility of the design in fulfilling its intended purpose. 
\end{comment}

This paper introduces a model-based approach for joint human-computer teaming in remote monitoring 
that strives to manage and mitigate some aspects of the inherent risk through formal design verification. The contribution to functional integration is a computationally independent model, \cite{garrido}, of the \emph{cognitive work problem} (CWP) that a system must solve \cite{workflowmodel,workcentered,BERRY201615,chi2010}.  A CWP is a complex \emph{object of work} that can be shared by the activities of a distributed cognitive system to transform it from some initial state to the required goal state. The technical neutrality of this object makes it sharable by actors whose performance properties are vastly different; thus, the CWP provides a common basis to represent its state changes that can be performed by a joint team of people and computers. 

The contribution to design verification then re-uses a CWP as a new form of evaluation criterion for model-checking to prove the correctness of the resulting integrated system designs. The capability to integrate human cognition with computer reasoning and then verify the design correctness increases the safety for a wide range of critical systems where human well-being is jeopardized by leaving human-computing teaming to informal approaches \cite{remote,Aalam229}.

This CWP-augmented workflow modeling approach is presented here in a case study of Bionous \phware. \phware is a tele-health application under development to enhance the safety of COVID-19 patients in home care with remote monitoring by clinicians. It provides a wearable ring or finger-clip to sense seven vital signs and send them to a smartphone via Bluetooth. A smartphone app uploads the sensor data to AI cloud services for storage and machine-learning to assign values to vitals, analyze their trends, and send alerts \cite{Altschul2004PredictiveMI,10.2307/2984877,10.5555/1643031.1643047}. A web-based dashboard lets clinicians review vitals' trends and alerts to decide appropriate interventions.

The paper details the CWP for \phware, developed in conjunction with clinicians and other experts,\footnote{Ann Marie Kimball, an internationally recognized MD epidemiologist in emerging infectious disease, provided medical advice} that defines actionable risk awareness of patients as a finite state diagram with two dimensions: physiological events and actions for appropriate care of the patient. This paper also details the translation of the CWP model into a suite of \emph{Linear Temporal Logic} (LTL) properties that together exactly capture the meaning of the CWP. LTL is a first-order predicate calculus that includes the ability to reason about the temporal ordering of events \cite{10.5555/975331}. Any viable design for vital sign remote monitoring of COVID-19 patients must verify against each LTL property created by the CWP in order to claim it takes appropriate action relative to patient risk.

The paper then describes the \phware\ workflow model. The model includes workflows for clinicians, AI cloud services, and patient-caregivers. It details how the workflow models for \phware\ are translated into equivalent \emph{Process Meta-language} (Promela) models for the SPIN model checker \cite{spin}. This translation includes behavior models for the environment inputs such as patient vitals. The paper reports the SPIN verification results. All the reported artifacts are available for independent review \cite{repo}.

The intentions of such a detailed case study are to show the viability of CWP augmented workflow modeling with model checking and to motivate the need for automated reasoning in any design environment. SPIN worked really well in this application in terms of performance and iterative design analysis. The workflows and CWP revisions were easy to verify when posted, and when SPIN found violations, the accompanying counter-examples provided critical insights to the unexpected, and often overlooked, outcomes common in asynchronous interactions. These insights were absolutely necessary for those working on the workflows and those working on the CWP to arrive at the presented final forms. The experience of this integrated design process with the model checker makes clear that it is difficult to reliably reason about even simple concurrent systems without a verification tool.