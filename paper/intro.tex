This paper introduces a new, model-based design method for interactive health information technology (IT) systems. This method extends workflow models with models of conceptual work products. When the health care work being modeled is substantially cognitive, tacit, and complex in nature, graphical work- flow models can become too complex to be useful to designers. Conceptual models complement and sim- plify workflows by providing an explicit specification for the information product they must produce. We illustrate how conceptual work products can be modeled using standard software modeling language, which allows them to provide fundamental requirements for what the workflow must accomplish and the information that a new system should provide. Developers can use these specifications to envision how health IT could enable an effective cognitive strategy as a workflow with precise information requirements. We illustrate the new method with a study conducted in an outpatient multiple sclerosis (MS) clinic. This study shows specifically how the different phases of the method can be carried out, how the method allows for iteration across phases, and how the method generated a health IT design for case management of MS that is efficient and easy to use.

We use the term conceptual work product (CWP) as the specification of a complex object of work that is shared by the activities in a distributed cognitive system, and transformed by them to its goal state. For brevity, we use CWP to mean specifications that include the complex object, its arriving state that is input to the system, the goal state that is the (output) product of the system, and any required intermediate states that must be satisfied before reaching the goal. These products are an important part of the domains surrounding their systems, and producing them is a fundamental purpose of a distributed cognitive system. Highly technical, critical examples include- 
\begin{compactitem}
\item medical diagnoses 
\item operational orders
\item plans for safety-critical operations 
\item schedules for scarce equipment
\item dynamic management of plans
\item classification of security events
\end{compactitem}
CWP specifications resemble templates or frames for stereotypical situations in artificial intelligence [2]. They also resemble the preconditions and post-conditions similar to those of a Hoare triple [3]. CWP share a focus on goal-satisfaction with several other techniques for requirements engineering [4]. 
A CWP is declarative representation of knowledge about the nature of a cognitive work problem [5]. Rummelhart & Norman [6] were among the first to report scientific evidence for the psychological distinction between declarative knowledge about the nature of something versus procedural knowledge of how to accomplish it. The declarative representation of the CWP can be defined in a manner that is not dependent on any particular technology, process, or system design. Thus, a CWP can serve as a primary system requirement that specifies what must happen without requiring any design decision about how it will happen [7; 8].

This paper is a case study applying model checking to a workflow model that claims to solve a CWP. The CWP is translated to an equivalent temporal logic specification. The workflow is translated to an equivalent Promela model. The SPIN model checker proves the workflow is a model of the CWP meaning that from the initial state of the system, it only resides in states allowed by the CWP, it only transitions between states allowed by the CWP, and it does this until the patient is either discharged from hame care or expires.

