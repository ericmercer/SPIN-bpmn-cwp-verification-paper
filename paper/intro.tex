There are limited tools for remote objective patient monitoring of diseases where patients may deteriorate quickly. Although tele-health is gaining acceptance for improving accessibility and reducing transmission rates \cite{10.1093/jamia/ocaa048,telehealth,10.1093/jamia/ocaa067}, the remote and asynchronous context of remote monitoring tele-health introduces the risk of a patient's conditions deteriorating before their physicians can be aware and intervene \cite{10.1097/ALN.0000000000003578}. Such risk always exists so the goal is to mitigate and manage that risk to enhance patient safety while in remote care. 

Remote patient monitoring relies on clinicians, health IT, patients-caregivers, and other concurrent actors to each reliably perform various asynchronous tasks to coordinate patient care and safety \cite{remote,Aalam229}. Designing such systems becomes complex quickly because of this dependency on distributed actors that perform outside the direct control of the system. As such, manually reasoning about functional integration and safety early in the design process is not obvious or easy; and yet, early in the design is exactly the time to clearly establish the utility of the design in fulfilling its intended purpose. 

This paper introduces a new model-based design approach for interactive health IT for remote tele-health that strives to manage and mitigate some aspects of the inherent risk through formal verification of the design. The approach extends workflow models with models of \emph{cognitive work problems} (CWPs) \cite{workflowmodel,workcentered,BERRY201615,chi2010}. When the health care work being modeled is substantially cognitive, tacit, and complex in nature, as in tele-health for remote patient monitoring, models, and especially graphical workflow models, are too complex for manual reasoning by designers. Conceptual models complement and simplify such workflows by providing an explicit specification for the cognitive problem they must solve. Formal verification can then be applied to the workflow models to automatically prove whether or not the workflows solve the CWP thereby providing designers with a mechanized automatic way to prove whether or not a design accomplishes its intended purpose early in the design process when change is less costly and intrusive.

The approach is presented by way of an example. The example details a workflow model to enable COVID-19 patients to care for themselves at home while being remotely monitored by clinicians. The system includes workflows for clinicians, AI cloud services, and patient-caregivers. The CWP for the system states the criteria for quantifying patient risk, and it states what actions must be taken in regards to different risk levels in order to enhance patient safety while in home care. 

A novel contribution in this paper is the translation of the CWP model into a suite of \emph{Linear Temporal Logic} (LTL) properties that together exactly capture the meaning of the CWP. LTL is a first-order predicate calculus that includes the ability to reason about the temporal ordering of events \cite{10.5555/975331}. Any viable design for home care must verify against each LTL property created by the CWP in order to claim it takes appropriate action relative to measured patient risk.

This paper then details how the workflow models for the example system are translated into equivalent \emph{Process Meta-language} (Promela) models for the SPIN model checker \cite{spin}. Behavior models are described for the environment inputs such as patient vitals, and the paper then presents the results of the verification by the SPIN model checker. 

The intent of such a detailed case study is to first show the viability of a model-based design method that includes the formal verification of the design against a declarative statement of what the design must accomplish in the form of a CWP; second, to motivate the need for automated reasoning in any design environment; and third, to make clear the process for applying the approach to other designs.  

What is described in this paper are the workflow models and CWP in their final forms. What is not presented due to space limitations are the many iterations on both that took place before arriving at the presented forms. The SPIN model checker found many instances where the workflow did not make appropriate decisions given the state of risk awareness, and there were many times where the system designer made changes to the workflow believing the changes to preserve correctness only to have the model checker show otherwise. At the same time, there were many instances where the model checker provided counter-examples showing how the CWP needed to change to meet the needs of risk based decision making. The experience of this integrated design process with the model checker makes clear that it is difficult to reliably reason about even simple concurrent systems without a verification tool.