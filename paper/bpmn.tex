The top green clinicians lane and the bottom salmon colored patient-caregiver lane in the workflow model in \figref{fig:bpmn} are directly modeled with a \emph{process} in Promela while the middle lane for the AI cloud server is implicitly modeled by the communication between the two processes. In other words, there is a non-deterministic choice in the patient/caregiver process at task 05 to set a value for \texttt{trndSvl}, and the Promela semantics, by definition, deliver that value at some point to the clinician process. Adding the extra process for the AI cloud server does not affect, or change, this explained basic behavior; it only serves to add non-necessary complexity.

The basic structure the for the clinicians and the patient-caregiver processes have the following structure.
%
{\small
\begin{lstlisting}[style=myPromela]
active proctype name() {
  /* Add tokens to initial elements */
  do
  :: (hasToken(element_0)) ->
    /* Model for element_0 activation */
  :: hasToken(element_1) ->
    /* Model for element_1 activation */
 
    /* activations for other elements */
    
  :: hasToken(element_n) ->
    /* Model for element_n activation */
  od
}
\end{lstlisting}
}
%
\noindent The Promela do-statement repeats infinitely often with each \emph{::} being a possible choice for the current iteration of the loop. The expression at each \emph{::} is a condition determining when a choice is available or not in the current iteration. In this way, the model checker must consider all combinations of active choices at each iteration of the loop, and thereby, resolve any and all interleaving of concurrent actions that may take place in the workflow associated with the process. The \texttt{(hasToken(element\_0))} expression has basis in the BPMN semantics.

The distributed and asynchronous BPMN semantics, as described in the standard, flow tokens through the workflow model to control activation of elements. As such, every visual construct defines activation conditions based on arrived tokens. Each element also defines what happens at activation in terms of consuming incoming tokens and generating outgoing tokens. The Promela model of the workflow follows these token based semantics. Each choice in the do-statement is associated with an element of the workflow. And the choice is active if the element meets its activation conditions based on tokens at its incoming edges.

For example, the diamond shaped gates in the workflow model are all exclusive-or gates as indicated by the small 'x' in the bottom corner. Here is the Promela implementation of the \texttt{Xor5 severity} gate on the left of the clinician lane.
%
{\small
\begin{lstlisting}[style=myPromela]
:: hasToken(clinicianXor5) ->
  atomic {
    consumeToken(clinicianXor5)
    if
    :: isHospital(orders) ->
      putToken(clinicianTask03)
    :: isDischarge(orders) ->
      putToken(clinicianEnd61)
    :: isHomeCare(orders) ->
      putToken(clinicianTask02)
    fi
  }
\end{lstlisting}
}
% 
\noindent The \emph{::} denotes that it is one of several choices that may be available at any iteration of the do-loop in the clinicians process. The \texttt{clinicianXor5} is a global variable to represent the presence or absence of a token on the incoming edge of the gate: \texttt{bit clinicianXor5 = 0}. 


The Promela implementation for the workflow model in \figref{fig:bpmn} assumes that elements never need, or consume, more than one token from the same source. It also assumes that there is never more than one token from the same source ever pending. The first assumption holds trivially in the workflow. The second assumption merits some discussion later after introducing the Promela model for each element. 


Exlusive-or means that only one token comes in at a time and goes out at a time on the gate.

The BPMN semantics for these gates under the state assumptions is that 


Th
Task 06 in the middle of the AI cloud server lane sends vitals to the clinician's lane. The frequency with which the clinician reviews vitals at the \emph{seconds catch AI} element in the middle of the top green clinician's lane is not at all coupled, correlated, or synchronized with task 06. That means that an arbitrary number of vitals could be sent between reviews. Ignoring the number of reported vitals does not affect the choices available to the clinician in terms of ordering an urgent exam.



It is also worth noting that time in not modeled in the Promela verification model. That means that any element of the model may undergo an arbitrary, possibly unbounded, delay. For example, the model checker may simply delay, forever, the clinician reviewing vitals if the patient is able to continue, forever, to generate vitals. SPIN implements a distributed semantics that intuitively means that any concurrent element can be delayed as long as other elements are able to make progress. In this way, the model checker considers any and all interleaving interactions between concurrent elements. As mentioned previously, a fairness constraint is used to exclude some infinite behavior such as the clinician ignoring forever the vitals from the patient.



These assumptions are restrictions on the general BPMN semantics but do not affect the model here because only one active patient is considered in the model and where tokens do accumulate, simple ignoring the number of tokens does not exclude any allowable behaviors. 

, and the one place where token 

Each construct used in the \phware\ workflow model in \figref{fig:bpmn} is discussed.